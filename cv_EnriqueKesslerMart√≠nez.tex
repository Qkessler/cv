%%%%%%%%%%%%%%%%%
% This is an sample CV template created using altacv.cls
% (v1.1.5, 1 December 2018) written by LianTze Lim (liantze@gmail.com). Now compiles with pdfLaTeX, XeLaTeX and LuaLaTeX.
%
%% It may be distributed and/or modified under the
%% conditions of the LaTeX Project Public License, either version 1.3
%% of this license or (at your option) any later version.
%% The latest version of this license is in
%%    http://www.latex-project.org/lppl.txt
%% and version 1.3 or later is part of all distributions of LaTeX
%% version 2003/12/01 or later.
%%%%%%%%%%%%%%%%

%% If you need to pass whatever options to xcolor
\PassOptionsToPackage{dvipsnames}{xcolor}

%% If you are using \orcid or academicons
%% icons, make sure you have the academicons
%% option here, and compile with XeLaTeX
%% or LuaLaTeX.
% \documentclass[10pt,a4paper,academicons]{altacv}

%% Use the "normalphoto" option if you want a normal photo instead of cropped to a circle
% \documentclass[10pt,a4paper,normalphoto]{altacv}

\documentclass[10pt,a4paper,ragged2e]{altacv}

%% AltaCV uses the fontawesome and academicon fonts
%% and packages.
%% See texdoc.net/pkg/fontawecome and http://texdoc.net/pkg/academicons for full list of symbols. You MUST compile with XeLaTeX or LuaLaTeX if you want to use academicons.

% Change the page layout if you need to
\geometry{left=1cm,right=9cm,marginparwidth=6.8cm,marginparsep=1.2cm,top=1.25cm,bottom=1.25cm}

% Change the font if you want to, depending on whether
% you're using pdflatex or xelatex/lualatex
\ifxetexorluatex
  % If using xelatex or lualatex:
  \setmainfont{Carlito}
\else
  % If using pdflatex:
  \usepackage[utf8]{inputenc}
  \usepackage[T1]{fontenc}
  \usepackage[default]{lato}
\fi

\usepackage{hyperref}
\catcode30=12
% Change the colours if you want to
%
\definecolor{Mulberry}{HTML}{72243D}
\definecolor{SlateGrey}{HTML}{2E2E2E}
\definecolor{LightGrey}{HTML}{666666}
\definecolor{links}{HTML}{A37785}
\colorlet{heading}{Sepia}
\colorlet{accent}{Mulberry}
\colorlet{emphasis}{SlateGrey}
\colorlet{body}{LightGrey}

% Change the bullets for itemize and rating marker
% for \cvskill if you want to
\renewcommand{\itemmarker}{{\small\textbullet}}
\renewcommand{\ratingmarker}{\faCircle}

%% sample.bib contains your publications
%\addbibresource{sample.bib}

% Defining the link styling.
\hypersetup{
    colorlinks=true,
    urlcolor=links,
  }
  
\begin{document}
\name{Enrique Kessler Martínez}
\tagline{Software Engineer}
%\photo{2.8cm}{}
\personalinfo{%
  % Not all of these are required!
  % You can add your own with \printinfo{symbol}{detail}
  \email{\href{mailto:enrique.kesslerm@gmail.com}{enrique.kesslerm@gmail.com}}
  \phone{+34 629580643}
  \homepage{\href{https://enriquekesslerm.com}{enriquekesslerm.com}}
  \linkedin{\href{https:://www.linkedin.com/in/enrique-kessler-martinez}{/enrique-kessler-martinez}}
  \github{\href{https://github.com/Qkessler}{/Qkessler}}
  %% You MUST add the academicons option to \documentclass, then compile with LuaLaTeX or XeLaTeX, if you want to use \orcid or other academicons commands.
  % \orcid{orcid.org/0000-0000-0000-0000}
}

%% Make the header extend all the way to the right, if you want.
\begin{fullwidth}
\makecvheader
\end{fullwidth}

%% Depending on your tastes, you may want to make fonts of itemize environments slightly smaller
% \AtBeginEnvironment{itemize}{\small}

%% Provide the file name containing the sidebar contents as an optional parameter to \cvsection.
%% You can always just use \marginpar{...} if you do
%% not need to align the top of the contents to any
%% \cvsection title in the "main" bar.
\cvsection[page1sidebar]{Experience}

\cvevent{Technology and Computer Sales}{El Corte Inglés}{Dec 2019 -- June 2020}{Cartagena, Murcia}
\begin{itemize}
\item Customer-facing.
\item Team with \textbf{Microsoft Teams} workflow.
\item \textbf{Contributed} to reach second place on the list of teams with the most sales.
\end{itemize}

% \divider

\cvsection{Projects}

\cvevent{enriquekesslerm.com}{Lead developer}{Mar 2021 -- present}{}
Using \textbf{Gatsby}, \textbf{Theme-UI} and deploying to \textbf{Gatsby Cloud}, I got my personal
web up and running. You can find more information about the functionality
implemented in the README file.

\vspace{2mm}

\textbf{Tech Stack}:
\cvtag{Gatsby}
\cvtag{Gatsby Cloud}
\cvtag{React}
\cvtag{HTML5}
\cvtag{CSS3}
\cvtag{JavaScript}

\vspace{2mm}

\github{\href{https://github.com/Qkessler/enriquekesslerm.com}{/enriquekesslerm.com}}
\homepage{\href{https://enriquekesslerm.com}{enriquekesslerm.com}}

\divider

\cvevent{AppMusic}{Project manager and Developer}{Sep 2020 -- Jan 2021}{}

AppMusic is a \textbf{Maven} Desktop App created with functionality that allows
for xml importing, and implemented with
\href{https://github.com/Qkessler/AppMusic#patterns}{patterns} from
the \textbf{GoF Design Patterns} book. Persistence was implemented with the
\href{https://www.h2database.com/html/main.html}{H2 database engine}.

\vspace{2mm}

\textbf{Tech Stack}:
\cvtag{H2 database engine}
\cvtag{Swing}
\cvtag{Java 11}\\
\cvtag{GoF design patterns}

\vspace{2mm}

\github{\href{https://github.com/Qkessler/AppMusic}{/AppMusic}}

\divider

\cvevent{CloudQuestions web}{Lead developer}{Apr 2020 -- Sep 2020}{}
CloudQuestions is a \textbf{Django} Web-App for studying which creates an easier customer experience applying a mix between “Flash Cards” and “Active Recall” studying techniques. It was deployed to \textbf{Google Cloud}, configuring instances (Ubuntu 20.04) and DNS.

\vspace{2mm}

\textbf{Tech Stack}:
\cvtag{Python3}
\cvtag{Django}
\cvtag{SQLite3}
\cvtag{PostgreSQL}
\cvtag{HTML5}
\cvtag{CSS3}
\cvtag{Twitter API}
\cvtag{Facebook API}
\cvtag{Google Calendar API}\\
\cvtag{Git Flow}

\vspace{2mm}

\github{\href{https://github.com/Qkessler/CloudQuestions}{/CloudQuestions}}
\homepage{\href{https://www.cloudquestions.es}{www.cloudquestions.es}}

\medskip

\clearpage

%% If the NEXT page doesn't start with a \cvsection but you'd
%% still like to add a sidebar, then use this command on THIS
%% page to add it. The optional argument lets you pull up the
%% sidebar a bit so that it looks aligned with the top of the
%% main column.
% \addnextpagesidebar[-1ex]{page3sidebar}


\end{document}

%%% Local Variables:
%%% mode: latex
%%% TeX-master: t
%%% End:
